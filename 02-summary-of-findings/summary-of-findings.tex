\chapter*{Summary of findings}
\addcontentsline{toc}{chapter}{Summary of findings}

During the study of the scenario conducted following the SecRAM2.0 methodology\cite{article:SecRAM}, a satisfying number of assets were analyzed. In particular, it became apparent that multiple physical and technical vulnerabilities were left untreated.
More specifically, there was a lack of documentation regarding the \texttt{Diginetwerk} private network, the \texttt{Citrix} virtualization infrastructure, and both the first-party and third-party authentication services. 

For all of these assets, sets of threats and vulnerabilities were provided. These sets included infrastructural, software, and configuration vulnerabilities.
Regarding \texttt{Diginetwerk}, we found that it was exposed to availability attacks like DDoS and Coremelt, but also there were no mechanisms in place to prevent router crashes, downtimes, and other technical issues.

For \texttt{Citrix}, the risk of hyperjacking, ransomware, and server crashes was discussed; while for the authentication services, the eventuality of password attacks, equipment tampering, and data leaks was taken into consideration.
Also, natural disasters and purposeful damages to the equipment were analyzed.

To reduce the impact and likelihood of a given incident a number of pre and post-incident controls have been proposed. Since this infrastructure is used for a time limited to the one of the elections, we tried to propose a set of moderately costly solutions, avoiding the adoption of full-scale disaster recovery sites. These proposals range from configuration testing to the adoption of physical security and DDoS prevention services.



